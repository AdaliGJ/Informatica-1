\documentclass{article}
\usepackage{fancyhdr} % Required for custom headers
\usepackage{lastpage} % Required to determine the last page for the footer
\usepackage{extramarks} % Required for headers and footers
\usepackage[usenames,dvipsnames]{color} % Required for custom colors
\usepackage{graphicx} % Required to insert images
\usepackage{listings} % Required for insertion of code
\usepackage{courier} % Required for the courier font
\usepackage{multirow}
\usepackage{hyperref}
\usepackage{amsmath}
\usepackage{amssymb}
\usepackage[utf8]{inputenc}

% Margins
\topmargin=-0.45in
\evensidemargin=0in
\oddsidemargin=0in
\textwidth=6.5in
\textheight=9.0in
\headsep=0.25in

\linespread{1.1} % Line spacing

%----------------------------------------------------------------------------------------
%	CODE INCLUSION CONFIGURATION
%----------------------------------------------------------------------------------------

\definecolor{MyDarkGreen}{rgb}{0.0,0.4,0.0} % This is the color used for comments
\lstloadlanguages{c} % Load Perl syntax for listings, for a list of other languages supported see: ftp://ftp.tex.ac.uk/tex-archive/macros/latex/contrib/listings/listings.pdf
\lstset{language=[sharp]c, % Use Perl in this example
        frame=single, % Single frame around code
        basicstyle=\small\ttfamily, % Use small true type font
        keywordstyle=[1]\color{Blue}\bf, % Perl functions bold and blue
        keywordstyle=[2]\color{Purple}, % Perl function arguments purple
        keywordstyle=[3]\color{Blue}\underbar, % Custom functions underlined and blue
        identifierstyle=, % Nothing special about identifiers                                         
        commentstyle=\usefont{T1}{pcr}{m}{sl}\color{MyDarkGreen}\small, % Comments small dark green courier font
        stringstyle=\color{Purple}, % Strings are purple
        showstringspaces=false, % Don't put marks in string spaces
        tabsize=5, % 5 spaces per tab
        %
        % Put standard Perl functions not included in the default language here
        morekeywords={rand},
        %
        % Put Perl function parameters here
        morekeywords=[2]{on, off, interp},
        %
        % Put user defined functions here
        morekeywords=[3]{test},
       	%
        morecomment=[l][\color{Blue}]{...}, % Line continuation (...) like blue comment
        numbers=left, % Line numbers on left
        firstnumber=1, % Line numbers start with line 1
        numberstyle=\tiny\color{Blue}, % Line numbers are blue and small
        stepnumber=5 % Line numbers go in steps of 5
}

\newcommand{\horrule}[1]{\rule{\linewidth}{#1}}

% Creates a new command to include a perl script, the first parameter is the filename of the script (without .pl), the second parameter is the caption
\newcommand{\perlscript}[2]{
\begin{itemize}
\item[]\lstinputlisting[caption=#2,label=#1]{#1.cs}
\end{itemize}
}

\begin{document}

\begin{tabular}{l l}
\multirow{5}{*}{\includegraphics[width=2cm]{escudo.png}} & Universidad del Istmo de Guatemala \\
 & Facultad de Ingenieria \\
 & Ing. en Sistemas \\
 & Informatica 1 \\
 & Prof. Ernesto Rodriguez - \href{mailto:erodriguez@unis.edu.gt}{erodriguez@unis.edu.gt} \\
\end{tabular}
\\

\begin{center}
        \horrule{0.5pt}
        \huge{Corrección Examen Parcial \#1} \\
        \large{Samantha Rodas Chaluleu}\\
        \large{Adalí Garrán Jiménez}\\
        \large{Fecha de entrega: 20 de Agosto, 2019} \\
        \horrule{1pt}
\end{center}

\section*{Ejercicio \#1}
Demueustre las siguientes propiedades utilizando induccón. Puede hacer uso de la aritmetica para dichas demostraciones. Asegurese de indicar claramente el caso base, el caso inductivo, la hipotesis inductiva y cada paso del procedimiento.\\
\begin{itemize}
    \item $\forall n\geq 1.~2*n~es~par$\\
    Caso Base: n=1
\[
2(1)~es~par
\]
\[
R.~2~es~par.
\]\\
Caso Inductivo: n=n+1\\
Hipótesis induciva: 2*n es par\\
\[
2(n+1)~es~par
\]
\[
[2n]+2~es~par
\]
\begin{center}Por Hipótesis inductiva [2n] es par\\
$\Rightarrow 2~tiene~que~ser~par$\end{center}
\[
R.~2~es~par
\]
    \item $\forall n\geq 4.~2^n<n!,~donde~n!=1*2*3*...*(n-1)*n$\\
Caso Base: n=4\\
\[2^4<1*2*3*4\]
\[R.~16<24\]\\
Caso Inductivo n=n+1\\
Hipótesis inductiva: $\forall n \geq 4,2^n<n!$
\[2^{n+1}<(n+1)!\]
\[2^n*2<n!*(n+1)\]
\[2*[2^n<n!]*(n+1)\]
\begin{center}Asumiendo por hipótesis inductiva que $[2^n<n!]$ es cierto:\end{center}
\[\Rightarrow Ej.\ 2 \ast 2^{5} < 5! (6)\ \ 64<720 \ (cumple)\]
\[2\leq(n+1)\]
\[1\leq n\]
\[n\geq 4 \Rightarrow n\geq 1\]
\[Cumple\]
\end{itemize}
\section*{Ejercicio \#2}
Dar una definicíon inductiva para las siguientes funciones sobre los numeros naturales unarios. Consejo, se le recomienda definir y utilizar la suma y multiplicación de los numeros naturales unarios:
\begin{itemize}
    \item La función facorial (n!):
    \[
        n! := \left\{
        \begin{array}{l l}
            1 & \mbox{si } n=o \\
            (x!)\otimes(\sigma(x)) & \mbox{si } n=\sigma(x) \\
        \end{array}
        \right.
    \]
    \item La función resta ($\ominus$):
    \[
        a\ominus b := \left\{
        \begin{array}{l l}
            0 & \mbox{si } a\leq b \\
            a & \mbox{si } b=0 \\
            \sigma(x\ominus b) & \mbox{si } a=\sigma(x)~\wedge ~ a>b \\
        \end{array}
        \right.
    \]
    \item La función sumatoria ($\sum_i^n$):
    \[
        \sum_i^n := \left\{
        \begin{array}{l l}
            n & \mbox{si } n=i \\
            \sum_i^x\oplus \ \sigma(x) & \mbox{si } n=\sigma(x)\\
        \end{array}
        \right.
    \]
    \item La función exponente ($a^b$):
    \[
        a^b := \left\{
        \begin{array}{l l}
            1 & \mbox{si } b=0 \\
            0 & \mbox{si } a=0 \\
            a\otimes a^i& \mbox{si } b=\sigma(i)\\
        \end{array}
        \right.
    \]
\end{itemize}
\section*{Ejercicio \#3}
A continuacion se presenta la definicion de la suma y mutiplicacion de numeros unarios:
\[
        a\otimes b := \left\{
        \begin{array}{l l}
            0 & \mbox{si } a=o \\
            0 & \mbox{si } b=o \\
            a & \mbox{si } b=1 \\
            b & \mbox{si } a=1 \\
            b\oplus(x\otimes b) & \mbox{si } a=\sigma(x) \\
        \end{array}
        \right.
    \]
\[
        a\oplus b := \left\{
        \begin{array}{l l}
            a & \mbox{si } b=0 \\
            b & \mbox{si } a=0 \\
            \sigma (x\oplus b) & \mbox{si } a=\sigma(x) \\
        \end{array}
        \right.
    \]\\
Puede utilizar una definición alterna (pero equivalente) de la suma o multiplicación si lo desea. Recuerde de indicar claramente el caso base, el caso inductivo, la hipótesis inductiva y cada paso de la demostración.\\
$2=\sigma ( \sigma(0))$\\
Demostrar que $2\otimes a=a\oplus a$:\\
Caso Base: a=0\\
\[2\otimes 0 = 0\oplus 0 \]
2=b
\[b\otimes 0 = 0 \]
\[0 = 0 \]
Caso Inductivo: $a=\sigma(x)$\\
Hipótesis inductiva: $2\otimes x= x\oplus x$\\
 $a=\sigma(x)$\\
$b=\sigma(\sigma(0))$\\
\[\sigma(\sigma(0))\otimes \sigma(x) = \sigma(x) \oplus \sigma(x)\]
$a_1=\sigma(i)=b$\\
$i=\sigma(0)$\\
$b_1=\sigma(x)=a$\\
\[a_1 \otimes b_1 = \sigma(x) \oplus \sigma(x)\]
\[\sigma(i) \otimes b_1 = \sigma(x) \oplus \sigma(x)\]
\[b_1\oplus (i\otimes b_1) = \sigma(x \oplus \sigma(x))\]
\[\sigma(x)\oplus (\sigma(0)\otimes \sigma(x)) = \sigma(x \oplus \sigma(x))\]
\[\sigma(x)\oplus (1\otimes \sigma(x)) = \sigma(x \oplus \sigma(x))\]
\[\sigma(x)\oplus \sigma(x) = \sigma(x \oplus \sigma(x))\]
\[\sigma(x\oplus \sigma(x)) = \sigma(x \oplus \sigma(x))\]
k=$x\oplus \sigma(x)$
\[\sigma(k) = \sigma(k)\]
\end{document}
